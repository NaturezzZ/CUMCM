\documentclass[12pt,letterpaper]{article}
\usepackage[UTF8]{ctex}
\usepackage{fullpage}
\usepackage[top=2cm, bottom=4.5cm, left=2.5cm, right=2.5cm]{geometry}
\usepackage{amsmath,amsthm,amsfonts,amssymb,amscd}
\usepackage{lastpage}
\usepackage{enumerate}
\usepackage{fancyhdr}
\usepackage{mathrsfs}
\usepackage{xcolor}
\usepackage{graphicx}
\usepackage{listings}
\usepackage{hyperref}

\hypersetup{%
  colorlinks=true,
  linkcolor=blue,
  linkbordercolor={0 0 1}
}
\begin{document}
\section*{Problem1}
考虑最优解,我们选取的是使得球弹起最小的高度,即球与鼓之间的相对距离的最大值为0.4m,这个方案下球会弹起最小的高度,
因此这是一个最节省能量的方案。

我们在这里建立一个相对简单的模型,此模型中,球与鼓碰撞之后进行竖直上抛运动,当落下至初始位置的时候与鼓再次相碰,继而形成周期运动。鼓首先进行一段时间的竖直上抛运动,
随后在某一位置$y_{d1f}$开始进行恒加速度的匀减速直线运动,直至运动停止,随即用相同的加速度进行向上的匀加速直线运动,到达碰撞的位置,与球进行碰撞。
$$T = \frac{2v_0}{g}$$
\begin{enumerate}
    \item
    鼓竖直上抛的过程经过t1结束,则
    $$y_{d1f} = v_{1i}t1 - \frac{1}{2}g t_1^2$$
    $$v_{1f} = v_{2i} = v_{1i}-gt1$$
    \item
    鼓进入匀加速直线运动状态,经过t2回到初始状态,
    $$0 = y_{d1f} + v_{2i}t2 + \frac{1}{2}a t_2^2$$
    鼓与球之间的最大距离为$\Delta H = 0.4m$,
    $$\Delta H = (v_0 - v_{1i})t1 + \frac{(v_0 -v_{1i})^2}{2(g+a)}$$
    $$v_{2f} = v_{2i} + a_2 t_2$$
    $$T = t1 + t2$$
    \item
    考虑碰撞的过程,设恢复系数为e,则考虑动量守恒与恢复系数的定义可以得到
    $$mv_0 + Mv_{1i} = Mv_{2f} - mv_{0}$$
    $$v_0 - v_{1i} = e(v_0 + v_{2f})$$
\end{enumerate}
我们根据一定的化简之后,使用MATLAB求其数值解,求解代码如附件Problem1.m所示。
我们不妨对于$e = ?$为例进行求解,
可得

\section*{Problem2}
首先计算转动惯量$I$
$$I = \iint dm(y^2 + z^2) = \iint  rd\theta dz (z^2 + r^2 \sin{\theta}^2) = 0.0865 kg\cdot m^2$$
对于2的情况,我们首先考虑只有一个力的存在时,对于鼓的质心所产生的力矩
$$\vec{M} = \vec{r}\times \vec{F}$$
$$|\vec{M}| = rF\alpha = 1.035 N\cdot m$$
根据角动量定理,此时的角加速度为
$$\beta = \frac{M}{I} = 11.97 s^{-2}$$
假设其匀角加速度运动,在0.1s内可以转过的角度为
$$\Theta = \frac{1}{2}\beta \Delta t^2 = 0.06 rad$$
这已经是本问题中所遇到的最大的旋转角,可以见得其依然很小,由于鼓面转动的角度很小,又由于鼓的半径与绳子的长度相比很小,我们便可以忽略在这个过程中,
力的方向的变化,继而认为这个过程中,力的方向是不变的。

产生误差可以分成两种情况,对称的情况与非对称的情况。对称的情况意味着早拉绳与拉力不同所带来的影响是同方向的,即可以认为鼓
围绕某一轴做定轴转动,非对称的情况即为拉绳起始时间与拉力不同所造成的影响方向不同,理论上应该研究刚体的定点转动。

我们首先研究对称的情况,

不妨根据定轴转动的轴对于不同的受力点进行编号,此处定轴转动的轴即为拉力对称轴通过圆心的垂线。利用已经求得的转动惯量,可以列出动力学方程如下
$$|\vec{M}| = I\frac{d\omega}{dt}$$
不妨设鼓面转过角度为$\delta$(虽然以上已经证明了角度转动很小,但是如下为了得到更加准确的结果,我们仍然考虑不同位置的不同$\delta$的情况)
$$\omega = \frac{d\delta}{dt}$$
% 这里可以描述一下怎么编号的
$$\vec{r} = (r\sin{\theta}\cos{\delta}, r\cos{\theta}\cos{\delta}, r\sin{\delta})$$
$$\vec{F} = (F\sin{\theta}\cos{\alpha}, F\cos{\theta}\cos{\alpha}, F\sin{\alpha})$$
$$\vec{M} = \vec{r} \times \vec{F}$$
这里我们使用将时间分割成小时间段的方法进行数值模拟,即利用
$$\omega = \omega + \frac{|M|}{I} \Delta t$$
$$\delta = \delta + \omega \Delta t$$
只需要求出每个位置的总力矩,就可以对其进行数值模拟,代码如附件Problem2.m所示
\end{document}